\documentclass[a4paper,10pt,twocolumn]{article} %{{{
%\documentclass[a4paper,10pt]{article} %{{{
\usepackage[utf8]{inputenc}

\usepackage{amsmath}
\usepackage{amsfonts}
\usepackage{amssymb}

\usepackage{epsfig} %for å lime inn filer

\usepackage{algorithmic}
\usepackage{algorithm}  
%\usepackage{babel}  

\newcommand{\ts}[1]{\textbf{#1}}
\newcommand{\bra}[1]{\langle{#1}|}
\newcommand{\ket}[1]{|#1\rangle{}}
%\newcommand{\braket}[1][2]{\langle{#1}|#2\rangle{}}
\newcommand{\abs}[1]{\left|{#1}\right|}
\newcommand{\expec}[1]{\langle{}{#1}\rangle{}}

%Define theorem enviroment. 
\newtheorem{theorem}{Theorem}[section]
\newtheorem{lemma}[theorem]{Lemma}
\newtheorem{proposition}[theorem]{Proposition}
\newtheorem{corollary}[theorem]{Corollary}
\newtheorem{assumption}[theorem]{Assumption}


\newenvironment{proof}[1][Proof]{\begin{trivlist}
\item[\hskip \labelsep {\bfseries #1}]}{\end{trivlist}}
\newenvironment{definition}[1][Definition]{\begin{trivlist}
\item[\hskip \labelsep {\bfseries #1}]}{\end{trivlist}}
\newenvironment{example}[1][Example]{\begin{trivlist}
\item[\hskip \labelsep {\bfseries #1}]}{\end{trivlist}}
\newenvironment{remark}[1][Remark]{\begin{trivlist}
\item[\hskip \labelsep {\bfseries #1}]}{\end{trivlist}}
\newcommand{\qed}{\nobreak \ifvmode \relax \else
      \ifdim\lastskip<1.5em \hskip-\lastskip
      \hskip1.5em plus0em minus0.5em \fi \nobreak
      \vrule height0.75em width0.5em depth0.25em\fi} %}}}

%opening
\title{ Project 1, Fys 4411 \\ }
\author{Karl R. Leikanger}


\begin{document}

\maketitle

\begin{abstract}%{{{

\end{abstract}%}}}

\section{Quantum dots}% {{{1

Our system of choice 
%}}}1

\section{The Monte Carlo method}%{{{1

\subsection{Introduction}%{{{2
Monte Carlo (MC) methods includes a wide range of computational methods for solving integrals. 
The main idea is to sample the integrand $f(\vec r)$ in random points $\{\vec r_i\}$, 
and finding an approximate solution to the integral by doing statistical analysis on the sampling data. 
The statistical error will give us an immediate idea of the precicion of the result.
%Statistical measures will give us an idea of the quality of the results

In general, any integral can be written on the form
\begin{align}
\int_{\mathcal D} f(\vec r) d\vec r = 
\int_{\mathcal D} P(\vec r) g(\vec r) d\vec r,
\end{align} 
where $P(\vec r)$ is a probability distribution. The integral can now be approximated by the sum 
\begin{align}
	\int_{\mathcal D} f(\vec r) d\vec r 
\approx \frac1N\sum_{i=1}^Ng(\vec r_i),
\end{align} 
where $\{\vec{r_i}\}$ is a set of random numbers drawn from the statistical distribution $P(\vec r)$.

The simplest example, possible whenever the region of the of the integral is finite,
is when $P$ is a uniform distribution bounded by the integration limits. Then
$P(\vec r)$ is a constant and $g(\vec r)=f(\ts r)$. However, for many integrals this method 
will converge slowly and will require a very large number of samples to
give accurate results. By choosing a smart $P(\vec r)$ that is large in the regions of interest, 
%for example where $g(\ts r)$ is large or changing rapidly,  
more samples will be drawn from the important regions, often leading to a more rapid convergence.  
%}}}2

\subsection{Markov Chain Monte Carlo}%{{{2

Markov Chain Monte Carlo (MCMC) provides us with a method for drawing the correct samples from $P$. 
Instead of drawing uncorrelated random samples we use a markow chain to generate a sequence $\{\vec r_n\}$ of vectors.
A Markov chain is caracterized by a transition kernel $q(\vec x|\vec y)$ which is a probabitity density distribution 
of the variable $\vec y$, and which tells us the probability of a transition from $\vec x$ to $\vec y$.
For each $\vec r_n$, we draw a new $\vec r_{n+1}$ from the distribution $q(\vec r_n,\vec y)$. The distribution 
of the vectors in the set 
$\{\vec r_n\}$ will converge towards the distribution $P$ if the following 
conditions are fulfilled (reference):
\begin{itemize}
\item
	$P$ must an invariant of the kernel $q(\vec x|\vec y)$ in the sense that 
	\begin{align}
	\int_{\mathcal D} d\vec r_A\, P(\vec r_A) q(\vec r_A | \vec r_B) = P(\vec r_B) \label{eq221},
	\end{align}
\item
	The chain must be ergodetic, which means that for ... . Can be expressed mathematically as ...
\end{itemize}

%More sophisticated methods exist, like the Metropolis-Hastings algorithm, where the drawing process is biased.
%To compensate for the biased 
%This method has the drawback that the samples are correlated which makes the analysis of the results somewhat more demanding. But the method also has
%some important advantages. One of them, as we will see later, is that the probabilityfunction $P$ does not need to be normalised. 
%This property comes very handy when dealing with many-particle wavefunctions in quantum mechanics.
%
%$P$ is sampled by generating a random new $\vec r_i$. The new position is then accepted by a probability 
%
%Both statements are actually quite intuitive, and i will now explain why(rewrite sentence):
It is costumary to make the analogy to a set of random (possibly biased) walkers, where
the density of walkers $P_w$ is stationary and proportional to $P$.
In the limit of an infinite number of walkers, 
the population of walkers in any point $\vec r_A$ must be constant,
\begin{align}
	\sum_n P_w(\vec r_A) q(\vec r_A|\vec r_n)  =	\sum_n P_w(\vec r_n) q(\vec r_n|\vec r_A) \label{eq222}
\end{align} 
This equation is equivalent to (\ref{eq221}) since both equaions states that the density is stationary.
%Of course, in a computer simulation the number of walkers is finite, and this equation can only be seen as an approximation.
%The markov chain must also be ergodetic. 
%One are free to choose any equation for the trajectories of the random walkers (any transition kernel) as long as (\ref{eq222})
The condition of detailed balance
\begin{align}
    P_w(\vec r_A) q(\vec r_A|\vec r_B)  =  P_w(\vec r_B) q(\vec r_B|\vec r_A).\label{detbal}
\end{align}
assures that (\ref{eq222}) is fulfilled. This equation is the basis for finding the transition 
kernels that is used in the metropolis-Hastings algorithm.
The markow chain must be ergodetic simply be.....
If not (ergodetic in regions: can be tackled ? )
%}}}2

\subsection{Metropolis-Hasting algorithm}%{{{2
We will now outline a general algorithm for the movement of the random walkers that fullfills (\ref{detbal}).
We introduce new scalar field to our equation $\gamma(\vec r_A,\vec r_B)$, which is a measure of the 
probability of accepting a move from $\vec r_A$ to $\vec r_B$.
We split our transition kernel in two parts
\begin{align}
	q(\vec r_A|\vec r_b) 
		\to 
	\gamma(\vec r_A|\vec r_b)
	g(\vec r_A|\vec r_b)
\end{align} 
We set $\gamma(\vec r_B|\vec r_A)=1$ when
\begin{align}
    &P_w(\vec r_A) g(\vec r_A|\vec r_B)  >  P_w(\vec r_B) g(\vec r_B|\vec r_A).
\end{align}
Then, by inserting this into (\ref{detbal}), we get
\begin{align} 
	\gamma(\vec r_A|\vec r_B)=max\left(
	\frac
	{P_w(\vec r_B) g(\vec r_B|\vec r_A)}
    {P_w(\vec r_A) g(\vec r_A|\vec r_B)}\, 
	,\, 1 \right)
\end{align}
By making this choice for $\gamma$, (\ref{detbal}) will be fulfilled and the sequence of $\{r_i\}$ will sample 
$P$. This rule preserves the relative provability of the transitions between $r_A$ and $r_B$, and therefore preserves the 
density of walkers in each point.  Note that only the relative ratio between the left and the right hand side of (eq) are of interest to us. Therefore, $P$ does not have to be normalized. 

These equations are implemented in the well known Metropolis-Hastings algorithm as follows:
\begin{algorithmic}
\LOOP{}
\STATE{$\vec r_B\gets$ random from distribution $g(\vec r_T|\vec r_B)$.}
\STATE{$\mathcal X\gets$ random uniform. ($\mathcal X\in[0,1]$)}%??(0,1)}
\STATE{Calculate $\gamma (\vec r_n|\vec r_T))$.}
\IF{$\mathcal X \le \gamma$}
\STATE{$\vec r_{n+1}\gets \vec r_T$}
\ELSE
\STATE{$\vec r_{n+1}\gets \vec r_n$}
\ENDIF{}
\STATE{$n\gets n+1$}
\ENDLOOP{}
\end{algorithmic}
Write some more about the algo..
Alternative algos, and their efficiency. refs.
%(more about the algo: convergence and thermalization (implementation issues))

%}}}2

\subsection{Choice of transition kernel} %{{{2

Brute force metropolis sampling: transition kernel $g(x|y)$ any distribution symmetric around x.\\
Metropolis-Hastings: $g(x|y)$, trajectories gouverned by Langevin dynamics. (trajectories of walkers: the solution to 
... equation).\\

Special choice of g, g=cymmetric around 0: gives the original metropolis algorithm.
Langevin eq and diffusion: not exact for dt>0, but gamma ensures that the sampling is correct. in the special case that 
dt = 0, gamma = 1 always, and every suggested move should be accepted. 

%}}}

\begin{figure}[h]
\begin{center}
	\includegraphics[width=8cm]{../datafiles/4000steps6partw1emin.eps}
\end{center}
\caption{{\it\small Example of the path of one walker during 4000 steps. The dotted line shows the path and the crosses 
marks the samplingpoints}}
\end{figure}

\begin{figure}[h]
\begin{center}
	\includegraphics[width=8cm]{../plots/1pd_2ptdt05emin2e6c.eps}
\end{center}
\caption{{\it\small Normalysized densityplot for 2 pt. $\alpha=0.98$, $\omega=0.4$, $\delta t=0.05$. $10^6$ cycles. }}
%\end{figure}

%\begin{figure}[h]
\begin{center}
	\includegraphics[width=8cm]{../plots/1pd6ptw1emin.eps}
\end{center}
\caption{{\it\small Normalysized densityplot for 6 pt. $\alpha=0.92$, $\omega=0.565$, $\delta t=0.05$. $10^6$ cycles. }}
%\end{figure}

%\begin{figure}[h]
\begin{center}
	\includegraphics[width=8cm]{../plots/1pd_12ptdt05emin2e6c.eps}
\end{center}
\caption{{\it\small Normalysized densityplot for 12 pt. $\alpha=0.87$, $\omega=0.68$, $\delta t=0.05$. $10^6$ cycles. }}
\end{figure}
We will outlining the Metropolis algorithm.

Marcow chains and random walkers.

Metropolis hastings.

%This method is referred to as importance sampling.

For many integrals
A possible problem doing importance sampling on the .. is to choose an .. that $P$ has to be normalized. In quantum monte carlo
the metropolis algorithm...

\subsection{The statistical analysis of the data}%{{{2

In our simulations, we will sample each point on the track of one single walker. 
This is possible as long as the system is ergodetic. That is, all $\vec r_n$ in $\mathcal D$ must be axessible to a 
random walker starting in any point $\vec r_0$ in $\mathcal D$. 
Otherwise, some parts of the integration domain would not be sampled.
After a certain number of moves, the statistical correlation between
the starting point $\vec r_0$ and the current position $\vec r_n$ will be very weak (there will be no). 
((reference, weaker statement))
If we denote the correlation time $\tau$ and $N$ is the total number of MCMC cycles, 
our simulation can equally well be seen as $\tau$ cycles of $\tau/N$ walkers
This set will converge to $P_w \propto P$.

A method to find the correlation time.
%}}}2


An more sophisticated way of sampling P is the Metropolis - Hastings algorithm.

\subsection{Quantum monte Carlo}%{{{2

In the Heisenberg image, the expectation values for any observable $\hat O$ can be expressed on the general form
\begin{align}
	\expec{ O(t) } = \int d\vec r \Psi(\vec r) \hat O(t) \Psi(\vec r)^*,
\end{align}
where $\Psi(\vec r) = \sum_i \psi_i(\vec r)$, for some basis $\{ \vec \psi_i \}$. 
%Generally, as long as we work in the heisenberg image and
%the wavefunctions are time independent, 		%we can always assume that the wavefunctions are real valued.
The integral can also be written 
\begin{align}
	\expec{ O(t) } = \int |\Psi(\vec r)|^2 \frac{ \hat O(t) \Psi(\vec r)^\dagger } {\Psi(\vec r)^\dagger},
\end{align}
Since $|\Psi(\vec r)|^2$ is a probability density distribution we see that
\begin{align}
	\expec{ O(t) } \approx \sum_{i=1}^N \frac{ \hat O(t) \Psi(\vec r_i)^\dagger } {\Psi(\vec r_i)^\dagger}, \label{RFQMC1} 
\end{align}
%
where the set of numbers $\{\vec r_i\}$ are drawn from $|\Psi|^2$.
This sum can be sampled using the metropolis algorithm with the metropolis ratio $|\Psi(r_i)|^2/|\Psi(r_{i+1})|^2$.
%
Note that $\Psi$ do not need to ne normalized in the above equations. This saves us a great deal of work since the normalization 
of most wavefunctions would have to be done numerically for every new trial function.% ????
%since any normalization constant will disappear both from the above fraction (\ref{RFQMC1}) and the metropolis ratio.

Show/argue that the variance is exactly 0 for the correct wf.

%next section
For our system, $\hat O\gets\hat H$, and $\{ \psi_i \}$ is the lowest energy eigenstates of the system. 
Since $[\hat H , \int dr \psi_i(\vec r)\psi_i(\vec r)]=0$, we can
assume that the the basis functions $\{ \psi_i \}$ are real valued without losing generality.

The quantum force:
We will assume that 



%}}}


%}}}1

\section{}%{{{1


%}}}1

\tableofcontents

\end{document}

% vim:foldmethod=marker

